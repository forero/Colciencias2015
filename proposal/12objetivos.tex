\section{Objetivos}

\subsection{Objetivos generales} 
% Enunciado que define de manera concreta el planteamiento del
% problema o necesidad y se inicia con un verbo en modo infinitivo, es
% medible, alcanzable y conlleva a una meta. 

Simular cat\'alogos de galaxias del Universo a gran escala que sigan
las condiciones realistas de observabilidad por instrumentos astron\'omicos. 
Para ello nos proponemos.


\begin{enumerate}
\item  Utilizar datos observacionales de la distribuci\'on de galaxias
  para encontrar relaciones sencillas entre la masa de halos de
  materia oscura y las  propiedades observables de una galaxia.
\item Crear un cat\'alogo ficticio de galaxias en escalas de $3$ Gpc$^3$
  a partir de simulaciones de materia oscura del Universo a gran escala.
\item Escribir rutinas que simulen el proceso de observaci\'on de
  estas galaxias con la instrumentaci\'on de DESI.
\item Crear un cat\'alogo ficticio de galaxias que incluyan los
  efectos observacionales de la instrumentaci\'on de DESI.
\end{enumerate}

El software generado por nuestro proyecto estar\'a a
disponibilidad  de la comunidad astron\'omica internacional a trav\'es
de repositorios p\'ublicos. 

\subsection{Objetivos espec\'ificos}
% Enunciados que dan cuenta de la secuencia lógica para alcanzar el
% objetivo general del proyecto. No debe confundirse con las
% actividades propuestas para dar alcance a los objetivos (ej. Tomar
% muestras en diferentes localidades de estudio); ni con el alcance de
% los productos esperados (ej. Formar un estudiante de maestría). 


\begin{enumerate}
\item Adquirir acceso a las simulaciones cosmol\'ogicas de materia
  oscura disponibles dentro de la colaboraci\'on DESI.

\item Analizar las anteriores simulaciones parar asignar luminosidades
  a los halos de materia oscura.

\item Encontrar los mejores par\'ametros que asignen luminosidad a
  materia oscura para reproducir las propiedades (funci\'on de
  correlaci\'on y distribuci\'on en redshift) de las galaxias que
  ser\'a observadas por DESI (LRG, ELG, QSO).

\item Construir cat\'alogos ficticios de galaxias (LRG, ELG, QSO) a
  partir de las simulaciones y los par\'ametros descritos
  anteriormente.

\item Escribir/adaptar software que describa la cadencia del
  telescopio DESI sobre el \'area del cielo a ser observada.

\item Escribir/adaptar software que describe la asignaci\'on de fibras
  \'opticas a las galaxias observables por DESI.

\item Escribir/adaptar sofware que describe la eficiencia de obtener
  una medici\'on confiable de redshift una vez ha sido apuntado con
  una fibra \'optica.

\item Escribir/adaptar software que simula el cat\'alogo de
  galaxias ficticias que tiene en cuenta la probabilidad de medir
  un redshift.

\item Escribir/adaptar sofware que vincula todos los pasos anteriors
  para simular de manera completa el experimento DESI hasta llegar a
  la creaci\'on de un cat\'alogo de galaxias que describe la
  estructura del Universo a gran escala.

\item Estudiar el impacto de la instrumentaci\'on de DESI en la
  estimaci\'on de par\'ametros cosmol\'ogicos a trav\'es de m\'etodos 
  de Oscilaciones Ac\'usticas de Bariones y de Distorsiones en el
  Espacio de Redshift.
\end{enumerate}


