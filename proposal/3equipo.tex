\section{Conformaci\'on del equipo de investigaci\'on}
%Colocar el nombre y código, registrado en el GrupLac, del o de los
%grupos de investigación. Al igual que el nombre de los demás
%integrantes que conforman el equipo de trabajo. Se debe incluir el
%tiempo de dedicación y funciones en el marco del proyecto.  

La investigaci\'on se har\'a dentro del grupo de astrof\'isica de la
Universidad de los Andes, c\'odigo GrupLAC COL0015473. 
Los integrantes del equipo para la actual propuesta son los
siguientes.  


\begin{itemize}
\item Un profesor de planta del grupo de Astrof\'isica de la
  Universidad de los Andes: Jaime Ernesto  Forero Romero, PhD.
  Dedicaci\'on: 7 horas/semana.
\item Dos estudiantes de doctorado en el departamento de F\'isica de
  la Universidad de los Andes.
\begin{itemize}
\item Felipe Leonardo G\'omez Cort\'es (F\'isico) con participaci\'on
  activa en los 36 meses del proyecto. Dedicaci\'on: 20 horas /
  semana.   
\item Estudiante por definir con participaci\'on activa en los 18 \'ultimos
  meses del proyecto. Dedicaci\'on: 20 horas / semana.  
\end{itemize}
\end{itemize}

\noindent

Con el apoyo de los siguientes asesores internacionales:

\begin{itemize}
\item Peder Norberg, PhD. Cient\'ifico en Durham University (Reino Unido).
\end{itemize}

Este proyecto es parte de las labores de la Universidad de los Andes
en la siguiente colaboraci\'on internacional.

\begin{itemize}
\item Colaboraci\'on internacional Dark Energy Spectroscopic Instrument
(DESI). 
\end{itemize}

\noindent
Adicionalmente el proyecto cuenta con el soporte del siguiente
personal t\'ecnico de la Universidad de los Andes:

\begin{itemize}
\item{Jheison Leonardo Rodr\'iguez. Ingeniero de c\'omputo del
  departamento de F\'isica.} 
\item{Juan Pablo Mallarino. F\'isico, coordinador del centro de de
  c\'omputo de alto rendimiento de la Universidad de los Andes.}
\end{itemize}
