
\section{Marco conceptual}
% Aspectos conceptuales y teóricos que contextualicen el problema de
% investigación en una temática; así como otros aspectos que sean
% pertinentes a juicio de los proponentes. 


El principal elemento conceptual de este proyecto es la {\bf Cosmolog\'ia
vista a partir de la estructura del Universo a gran escala}. 
Los principales aspectos de esta área de conocimiento se pueden
dividir en tres ramas: teoría, simulaciones y observaciones.


\subsection{Teor\'ia}

Desde el punto de vista te\'orico hay dos tipos de observables que nos
dan claves sobre los contenidos y características del Universo.
El primero es la historia de expansi\'on del Universo; el segundo es
la evolucíon de la distribución de galaxias a gran escala.
Ambos observables se pueden obtener a partir de mediciones de
posiciones de millones de galaxias.

La {\bf historia de expansi\'on} del Universo, parametrizada por la
constante de Hubble dependiente del redshift H(z), dependende
directamente del contenido de mater\'ia y energ\'ia del Universo. 
Este contenido se puede serparar en tres componentes: la densidad de materia
bari\'onica ($\Omega_b$), la densidad de materia oscura
($\Omega_{dm}$) y la densidd de energ\'ia  oscura  ($\Omega_{DE}$), la
cual puede variar en el tiempo dependiendo del valor de la constante
$w$ en la ecuaci\'on de estado, $\Omega_{DE}$ donde 

\begin{equation}
\Omega_{DE}(z) = \Omega_{DE,0}(1+z)^{3(1+w)},
\end{equation}
%
asumiendo que el Universo es plano
($\Omega_b+\Omega_{dm}+\Omega_{DE}=1$) y que $w$ es una constante. 
En esta ecuaci\'on tomamos a $\Omega_{DE}$ de manera general para una
sin asumir que se trata de una constante cosmol\'ogica,
$\Omega_\Lambda$ con $w=-1$.   

Actualemente estos par\'ametros cosmol\'ogicos es\'an
acotados observacionalmente cerca a los siguientes valores $\Omega_b=0.05$,
$\Omega_{dm}=0.25$, $\Omega_{DE}=0.70$ y $w=-1$. 
Pero hay mucho interés por medir con alta precisión
$\Omega_{DE}$ y $w$.
La razon es que estas dos cantidades est\'an directamente 
relacionadass con  la expansi\'on acelerada del Universo.
Explicaciones de esta aceleración pueden corresponder a una simple
constante cosmol\'ogica o pueden ser la evidencia de que Relatividad
General no es la teor\'ia correcta para la gravedad 
\cite{2014arXiv1401.0046M}.

Por otro lado, el {\bf crecimiento de estructura a gran escala} depende
principalmente de dos cosas: las fluctuaciones iniciales en el campo de densidad
de materia en las \'epocas tempranas del Universo y de la teor\'ia de
la gravedad que hace que esas fluctuaciones se amplifiquen.

En este contexto la variable relevante es el contraste de
densidad $\delta({\bf  x},t)\equiv\rho({\bf r},t)/\bar{\rho(r)}-1$ de
la materia oscura. 
En el r\'egimen lineal este crecimiento está descrito 
por la siguiente ecuaci\'on diferencial 

\begin{equation}
\ddot{D} + 2H(z)\dot{D}- \frac{3}{2}\Omega_mH_{0}^2(1+z)^3D=0,
\end{equation}
%
donde $H_0$ es la constante de Hubble en el presente y $D(t)$ es la
funci\'on de crecimiento. 
Aunque incluso en el r\'egimen no lineal el contraste de densidad
tambi\'en depende de esta funci\'on $D(t)$.   

Lo interesante es que diferentes modelos de la gravedad
(entre ellos la Relatividad General) hacen predicciones sobre el
comportamiento de esta funci\'on $D(t)$. 
De esta manera, una estrategia posible para probar teorias de gravedad
modificada es medir al mismo tiempo la historia de expansi\'on y la
historia de crecimiento de estructuras para ver si dan valores
consistentes de $H(z)$ y de $w$ \cite{2014arXiv1401.0046M}.  

En resumen, {\bf la teor\'ia predice que mediciones de la distribuci\'on de
galaxias tienen informaci\'on para, simultaneamente, poner a prueba la
teoria de la Relatividad General y medir la composici\'on del
Universo.} 

\subsection{Simulaciones}

La decripci\'on anal\'itica esbozada anteriormente
nos acercar a varios fen\'omenos del crecimiento de estructura.
Sin embargo, una descripci\'on detallada solamente es posible a
trav\'es de las simulaciones computacionales. 


En el caso de simulaciones para la cosmolog\'ia est\'andar $\Lambda$
Cold Dark Matter ($\Lambda$CDM) se sigue un aproximaci\'on en la cual
solamente se simula la componente oscura y no la luminosa, dado que
la primera domina la din\'amica del sistema y es suficiente para tener
una primera aproaximación sobre la distribución espacial de las
galaxias, el observable más importante para nuestro proyecto.

En estas simulaciones se toma un volumen computacional de un volumen
representativo del Universo.
La distribuci\'on de materia oscura es discretizada en part\'iculas
computacionales con posiciones y velocidades que pueden evolucionar en
el tiempo.  
Esta forma general de realizar los c\'alculos ha sido
perfeccionada durante los \'ultimos treinta a\~nos de investigaci\'on
en el tema de formaci\'on de estructura en un Universo domiando por la
materia oscura
\cite{1985ApJ...292..371D,1999ApJ...522...82K,2005Natur.435..629S}. 

Como en todo experimento num\'erico hay dos cantidades centrales para
la descripci\'on del sistema: las condiciones iniciales y las reglas
para su evoluci\'on temporal.
Las condiciones iniciales est\'an definidas por las posiciones y velocidades
iniciales de las part\'iculas computacionales. 
En este caso los desplazamientos de las part\'iculas
con respecto a una distribuci\'on homog\'ena de part\'iculas están
relacionadas con parámetros cosmológicos y el tipo de materia oscura
que se considera.
Una vez las posiciones iniciales est\'an determinadas se
determinan las velocidades iniciales, usualmente a trav\'es de la
aproximaci\'on de Zeldovich aunque es posible imponer estas
condiciones iniciales a trav\'es de otras aproximaciones perturbativas
\cite{2014MNRAS.439.3630W}. 

La simulaci\'on sigue entonces la evoluci\'on de las posiciones y
velocidades de las part\'iculas durante la historia del
Universo. En el rango de evoluci\'on no lineal se forman
sobre-densidades de materia oscura (halos) en
donde  las galaxias deberian formarse y evolucionar. 

En este punto es posible entonces pasar a una
descripci\'on de la distribuci\'on de materia en la simulaci\'on a
partir de las posiciones  velocidades y masas de estos {\bf halos de materia
  oscura}. 
Estos objetos serán los centrales al momento de vincular
las predicciones de los modelos con las observaciones. 
Estos halos sirven en las construcci\'on de {\bf cat\'alogos
  ficticios} que se son comparables con mapas de la distribuci\'on de
galaxias en el Universo una vez se ha decidido sobre un mapeo entre
halos de materia oscura y materia visible. 


\subsection{Observaciones}

Las observaciones del Universo a gran escala empezaron a tener un rol
central en la astronom\'ia observacional y en la cosmolog\'ia con dos
colaboraciones: el Sloan Digital Sky Survey (SDSS) \cite{SDSS} y el Two Degree
Field Galaxy Redshift Survey (2dFGRS) \cite{2dF}. 
El objetivo principal de estas campa\~nas observacionales fue tomar
espectros de cerca de un mill\'on de galaxias del Universo local sobre
una gr\'an \'area del cielo. 
A partir de estas observaciones de la estructura del Universo a gran
escala se pueden inferir diferentes par\'ametros cosmol\'ogicos.

Dentro de las mediciones m\'as importantes a partir de estas dos misiones
ha sido el de la densidad de materia cosmol\'ogica $\Omega_m$
\cite{2001Natur.410..169P} y el pico en la funci\'on de
correlaci\'on correspondiente a la Oscilacion Ac\'ustica de Bariones
(OAB) \cite{Eisenstein2005}. 
Observaciones recientes de la OAB son las que proveeen una de las
form\'as mas competitivas de acotar la historia de expansi\'on del
Universo \cite{2014MNRAS.441...24A}. 

En cuanto a las mediciones del crecimiento de estructura (cotas sobre
la forma de la funci\'on $D(t)$ mencionada en la secci\'on sobre
teor\'ia) estas se hacen a trav\'es de la cuantificaci\'on de la
distrbuci\'on de las posiciones de las galaxias en la direcci\'on
radial y perpendicular al observador \cite{2014MNRAS.439.3504S},
aunque los resultados actuales no tienen la precisión suficiente 
para distinguir entre el modelo est\'andar y teor\'ias
modificadas de la gravedad. 

DESI es una colaboraci\'on internacional que tiene como objetivo medir
el corrimiento al rojo de 25 millones de galaxias tomados de una lista
de targets identificados de manera espectrosc\'opica
Esto es 10 veces m\'as espectros de galaxias de los que han sido
tomados hasta la fecha.
Estos espectros ser\'an tomados con 5000 posicionadores autom\'aticos
cubriendo un \'area del cielo de 14000 grados cuadrados
aproximadamente. 
Se observar\'an principalmente tres tipos de galaxias, galaxias
luminososas rojas (LRG, por su iniciales en ing\'es), galaxias de
l\'ineas de emisi\'on (ELG) y cu\'asares (QSO) \cite{2015AAS...22533610C}. .
El survey se har\'a  en el observatorio de Kitt Peak en Arizona
(Estados Unidos) con el telescopio Mayal de 4 metros entre el 2018 y
el 2023. 
El objetivo principal ser\'a medir la historia de expansi\'on del
Universo con una precisi\'on sin precedentes usando este mapa
tridimensional de galaxias. 

{\bf En este proyecto nos proponemos producir simulaciones del
  Universo a gran escala tal como ser\'a observado con DESI.}




