
\section{Estado del arte}
% Revisión actual de la temática en el contexto nacional e
% internacional, avances, desarrollos y tendencias. 

\subsection{Pruebas observacionales de par\'ametros cosmol\'ogicos}
Con la puesta en marcha de ambiciosos programas observacionales para
cuantificar la Ener\'gia Oscura durante la siguiente d\'ecada, los
esfuerzos te\'oricos se han  multiplicado para encontrar formas de
medir la expansi\'on acelerada con alta precisi\'on. 
Existen varios
m\'etodos observacionales utilizados en la actualidad para hacer este
tipo de mediciones. 
Entre ellos se destacan:

\begin{enumerate}
\item La escala de la Oscilaci\'on Ac\'ustica de Bariones (OAB).
\item El crecimiento de estructura a trav\'es de las Distorsiones en
  el Espacio de Redshift (DER) o Redshift Space Distortions (RSD).
\item El bosque Lyman-$\alpha$ de cuásares distantes.
\item Efecto de lente gravitacional d\'ebil.
\item Mediciones de distancias con supernovas.
\item Abundancia de c\'umulos de galaxias.
\end{enumerate}

Para misiones observacionales desde tierra los m\'etodos m\'as interesantes 
son los primeros tres. 
En el caso de misiones espaciales los seis m\'etodos son considerados.

El más favorecido en ambos casos es el m\'etodo de OAB. 
Desde la primera detecci\'on en el 2005 \cite{Eisenstein2005},
ha sido uno del los preferidos para obtener mediciones precisas de la
historia de expansi\'on. Ejemplos recientes de estas mediciones
\cite{wigglez,BOSS} para
galaxias m\'as distantes han sido hechas por BOSS
(\url{http://www.sdss3.org/surveys/boss.php}) y WiggleZ
(\url{http://wigglez.swin.edu.au/site/}). 
Ambas son colaboraciones compuestas por cientos de cient\'ificos, dada
la gran cantidad de las observaciones, reducci\'on de datos y
simulaciones que deben ser hechas para obtener los resultados. 

Usando los mismos datos de las observaciones ha sido posible
medir las DER para poner cotas sobre la funci\'on $D(t)$ que describe
el crecimiento de estructura \cite{2014MNRAS.439.3504S}. 
Los resultados muestran consistencia con una cosmolog\'ia LCDM donde la
gravedad est\'a descrita por la Relatividad General, aunque los datos
dan una leve preferencia a valores donde la interacci\'on
gravitacional es levemente menor a las predicciones de la Relatividad
General. 

Otra v\'ia de nuestro inter\'es para acotar par\'ametros
cosmol\'ogicos son las pruebas novedosas que se benefician de los
grandes vol\'umenes que ser\'an observados.
Entre estos se encuentran 

\begin{itemize}
\item El efecto Alcock-Paczinsky sobre la estructura a gran escala \cite{2013arXiv1309.1162S}.
\item Las eventos de colisiones extremas, como el Bullet Cluster o el
  Bullet Group \cite{Bullets2010, Bullets2014}. 
\end{itemize}


\subsection{Campa\~nas observacionales}

El objetivo de campa\~nas observacionales que empiezan en $\sim 5$
a\~nos es observar al menos un orden de magnitud m\'as de galaxias en
un rango de redshift $0.7<z<2.0$ de las observadas hasta ahora,
extendiendo el rango actual $z<0.7$. 
Dentro de estas misiones se destacan:

\begin{itemize}
\item {\bf Dark Energy Spectroscopic Instrument (DESI)}. Empezar\'a a
  tomar datos en el 2018. Espera tomar espectros de cerca de +25 millones de
  galaxias utilizando el telescopio de 4 metros en el Kitt Peak
  Observatory en Estados Unidos \cite{DESI}.
\item {\bf 4MOST}. Empezar\'a a tomar datos en el 2019. Espera
  tomar espectros de cerca de +20 millones de estrellas galaxias con el
  telescopio de 4 metros VISTA en Chile. Su objetivo principal es
  hacer ciencia de la formaci\'on de la V\'ia L\'actea \cite{4MOST}.
\item {\bf Euclid}. Empezar\'a a tomar datos en el 2021. Ser\'a un
  telescopio espacial de 1.2 metros. Espera tomar espectros de cerca
  de 52 millones de galaxias. \cite{EUCLID}.
\end{itemize}

La ventaja de DESI sobre 4MOST es que estar\'a completamente dedicado
a observar galaxias; y sobre Euclid es que tomar\'a datos 3 a\~nos
antes. De esta manera DESI ofrece una oportunidad \'unica para hacer
cosmolog\'ia observacional en los siguientes a\~nos. 

DESI est\'a entre el 2014-2017 en la fase de planeaci\'on del survey y
construcci\'on del instrumento. {\bf Es justamente en esta fase en la que
este proyecto espera hacer sus contribuciones a DESI}.


\subsection{Simulaciones para el dise\~no de DESI}

En esta fase de dise\~no una de los objetivos principales es lograr la
simulaci\'on completa del  instrumento, algo que se conoce como {\bf
  end-to-end simulation}. Esto implica simular eventos como la
secuencia en la que el telescopio apuntar\'a al cielo, c\'omo se van a
ubicar las fibras \'opticas sobre las galaxias observadas, los
espectros de cada una de las galaxias observadas con ruido
instrumental realista, la toma de espectros sobre c\'amaras CCD, la
extracci\'on de la informaci\'on hasta la deducci\'on del redsfhit de
cada una de las galaxias observadas. 
Luego de esto se puede aplicar
cada m\'etodos (OAB, DER, etc) para ver con qu\'e precisi\'on se
pueden acotar los par\'ametros cosmol\'ogicos. 

Central a este esfuerzo es crear cat\'alogos de Universos ficticios a
partir de simulaciones de N-cuerpos. 
Esto permitir\'a hacer las
simulaciones end-to-end realistas y al mismo tiempo entrenar los
algoritmos que finalmente van a utilizar los datos observacionales
para medir los datos sobre la expansi\'on del Universo. El estado del
arte en estas simulaciones se compone por vol\'umenes c\'ubicos de
universos  de lado $1000-10000$\hMpc
con un n\'umero de part\'iculas entre los rangos
$1024^3$-$7120^3$ en una cosmolog\'ia est\'andar LCDM. Estas
simulaciones han sido hechas por diferentes grupos, entre ellas caben
destacar LASDAMAS en Stanford
(\url{http://lss.phy.vanderbilt.edu/lasdamas/team.html}), HorizonRun en KIAS (hecha por nuestro asesor Changbom Park y su grupo)
(\url{http://sdss.kias.re.kr/astro/Horizon-Run23/}) y MultiDark (hecha
por el equipo de nuestro asesor Stefan Gottloeber)
(\url{http://www.multidark.org/MultiDark/}). Tambi\'en cabe resaltar
la seri de de simulaciones CODECS que incluye simulaciones en
diferentes familias de energ\'ia oscura
(\url{http://www.marcobaldi.it/web/CoDECS_summary.html}). 

En todos estos casos es un objetivo de la comunidad tener más
volúmenes simulados, con diferentes códigos y diferentes técnicas. 
Esto permite que los resultados computacionales sean robustos y con
pocos sesgos por tener un número pequeño de volúmenes simulados. 
{\bf En este proyecto nos proponemos utilizar simulaciones de grandes
volúmenes del Universo para convertirlars en cat\'alogos ficticios de
galaxias que  luego puedan ser procesados a trav\'es del una serie de
programas que simulen las observaciones de DESI y estimar con qu\'e
precisi\'on pueden inferirse los par\'ametros cosmol\'ogicos de los
universos simulados.}











