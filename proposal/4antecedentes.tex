\section{Antecedentes y resultados previos del equipo de
  investigaci\'on} 
%  investigación solicitante en la temática específica del proyecto} 
% trayectoria del equipo de investigación con relación al problema
% planteado en el proyecto. 

El l\'ider del equipo de investigaci\'on (Prof. Forero Romero) tiene un PhD en f\'isica en el \'area de simulaciones de formarci\'on de galaxias en un  contexto
cosmol\'ogico, 5 a\~nos de experiencia postdoctoral en el tema (en
Alemania y Estados Unidos) y 3 a\~nos de trabajo como
investigador/profesor en el grupo de  Astrof\'isica de la Universidad
de los Andes. 
Sus investigaciones recientes en la Universidad de los Andes han
explorado diversos aspectos del modelamiento y caracterizaci\'on de la
estructura del Universo a gran  escala: pruebas cosmol\'ogicas
\cite{2014ApJ...796..137L}, descripci\'on del universo a gran escala
\cite{2014MNRAS.443.1090F,2013MNRAS.428.2489L,2012MNRAS.425.2049H},
acceso de datos de simulaciones computacionales a trav\'es de bases de
datos \cite{2013AN....334..691R} y planeaci\'on de surveys
espectrosc\'opicos de galaxias \cite{2014SPIE.9150E..23S}.
  

Durante el 2015 el Prof. Forero Romero gestin\'o la entrada de la
Universidad de los Andes en la colaboraci\'on internacional
DESI. (Dark Energy Spectroscopic Instrument) liderado por el Lawrence 
Berkeley National Laboratory en Estados Unidos
(\url{http://desi.lbl.gov/}). 
Esta colaboraci\'on medir\'a el efecto
de la energ\'ia oscura en la historia de expansi\'on del
Universo. 
DESI medir\'a los espectros de cerca de 20 millones de galaxias
para crear un mapa 3D del Universo con  profundidad de 10mil
millones de a\~nos luz. 
Dentro de esta colaboraci\'on se ha trabajado en conjunto al grupo de
simulaciones de la Durham Univeristy representado por el Dr. Peder
Norberg.  



En los \'ultimos 5 a\~nos, el lider del grupo se ha dedicado a
investigar y publicar en el tema. 

En los temas relevantes para esta propuesta sobresalen las siguientes 
presentaciones en la Am\'erican Astronomical Society:


\begin{itemize}
\item {\it The Dark Energy Spectroscopic Instrument (DESI): Tiling and
  Fiber Assignment} Cahn, Robert N.; Bailey, Stephen J.; Dawson, Kyle
  S.; {\bf Forero Romero, Jaime}; Schlegel, David J.; White, Martin;
  DESI Collaboration, American Astronomical Society, AAS Meeting \#225,
  \#336.10, (2015)
\item {\it The Dark Energy Spectroscopic Instrument (DESI): Data
  Systems} 	
	Bailey, Stephen; Bolton, Adam S.; Cahn, Robert N.; Dawson,
        Kyle; {\bf Forero Romero, Jaime}; Guy, Julien; Kisner, Theodore;
        Moustakas, John; Nugent, Peter E.; Schlegel, David J.; Stark,
        Casey; Weaver, Benjamin; DESI Collaboration , American
        Astronomical Society, AAS Meeting \#225, \#336.09, (2015)
\end{itemize}


Y la siguiente publicaci\'on en Proceedings de la International
Society for Opticans and Photonics 
\begin{itemize}
\item {\it Target allocation yields for massively multiplexed
  spectroscopic surveys with fibers} 	
	Saunders, Will; Smedley, Scott; Gillingham, Peter;
        {\bf Forero-Romero, Jaime E.}; Jouvel, Stephanie; Nord, Brian,
        Proceedings of the SPIE, Volume 9150, id. 915023 10
        pp. (2014). 
\end{itemize}


En el tema de estudios cosmol\'ogicos de formaci\'on de estructura del
Universo a gran escala.

\begin{itemize}

\item {\it Cosmological Constraints from the Redshift Dependence of
  the Alcock-Paczynski Test: Galaxy Density Gradient Field} 	
	Li, Xiao-Dong; Park, Changbom; {\bf Forero-Romero, J. E.}; Kim,
        Juhan, ApJ, 796, 2, 137, (2014)

\item {\it Cosmic web alignments with the shape, angular momentum and
  peculiar velocities of dark matter halos} {\bf Forero-Romero} J.E.,
  Contreras S., Padilla N., MNRAS, 443, 2, 1090, (2014)

\item{\it The velocity shear tensor: tracer of halo alignment},
  Libeskind N., Hoffman Y., {\bf Forero-Romero} J.E., Gottloeber S.,
  Knebe A., Steinmentz M., Klypin A., MNRAS 428, 2489, (2013) 

\item
{\it Halo based reconstruction of the cosmic mass density field}
Mun\~oz-Cuartas J. C., M\"uller V, {\bf Forero-Romero} J. E., MNRAS,
417, 1303, (2011)

\item
{\it A Dynamical Classification of the  Cosmic Web}.  {\bf
  Forero-Romero} J.E., Hoffman Y.,  Gottloeber S., Klypin A., Yepes G., MNRAS, 396, 1815-1824, (2009)

\end{itemize}


En el \'area de analizar datos de simulaciones cosmol\'ogicas
para hacerlas disponibles a la comunidad acad\'emica:

\begin{itemize}
\item{\it The MultiDark Database: Release of the Bolshoi and
  MultiDark Cosmological Simulations} , K. Riebe , A. M. Partl,
  H. Enke, {\bf J.E. Forero-Romero}, S. Gottloeber, A. Klypin,
  G. Lemson, F. Prada, J. R. Primack, M. Steinmetz, V. Turchaninov,
  Astronomische Nachrichten, 334, 691, (2013). 
\end{itemize}

Esta propuesta es la continuaci\'on del proyecto {\it{De la V\'ia
    L\'actea a las galaxias m\'as distantes}} apoyado por la
Vicerrector\'ia de Investigaciones de la Universidad de los Andes en
el per\'iodo 2012-2015. 
En el ap\'endice se encuentran todas las publicaciones (10 en total)
que fueron resultado de ese proyecto. 




