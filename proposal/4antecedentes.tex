\section{Antecedentes y resultados previos del equipo de
  investigaci\'on} 
%  investigación solicitante en la temática específica del proyecto} 
% trayectoria del equipo de investigación con relación al problema
% planteado en el proyecto. 

El l\'ider del equipo de investigaci\'on tiene un PhD en el \'area de
simulaciones de formarci\'on de galaxias en un  contexto
cosmol\'ogico, 5 a\~nos de experiencia postdoctoral en el tema (en
Alemania y Estados Unidos) y 2 a\~nos de trabajo como
investigador/profesor en el grupo de  Astrof\'isica de la Universidad
de los Andes. 


Adicionalmente durante el \'ultimo a\~no, se ha involucrado en dos
colaboraciones internacionales: 

\begin{itemize}
\item Con el experimento DESI
(Dark Energy Spectroscopic Instrument) liderado por el Lawrence
Berkeley National Laboratory en Estados Unidos
(\url{http://desi.lbl.gov/}). 
Esta colaboraci\'on medir\'a el efecto
de la energ\'ia oscura en la historia de expansi\'on del
Universo. 
DESI medir\'a los espectros de cerca de 20 millones de galaxias
para crear un mapa 3D del Universo con  profundidad de 10mil
millones de a\~nos luz. 
El contacto principal de Jaime Forero en esta colaboraci\'on desde
comienzos del 2014 es el Dr. Robert Cahn, con quien se han venido
desarrollando trabajos para la prepraci\'on del experimento que
empezar\'a a tomar datos en el 2018. 

\item Con el equipo de Cosmolog\'ia del Korean Institute for Advanced
  Studies (KIAS) (\url{http://www.kias.re.kr/}). 
Esta colaboraci\'on se centra en el desarrollo de nuevos 
  algoritmos para acotar los par\'ametros cosmol\'ogicos a trav\'es de
  mediciones de la distribuci\'on de galaxias a gran escala. 
  El contacto principal en esta colaboraci\'on desde mediados del 2013 es
  el Dr. Changbom Park. 
  Un aspecto atractivo de esta colaboraci\'on es
  la posibilidad de aplicar estos algoritmos sobre datos observacionales del
  SDSS-III (Sloan Digital  Sky Survey) (\url{https://sdss3.org/}) en
  el periodo en el que todav\'ia no son p\'ublicos los  datos, gracias
  al que KIAS hace parte de la colaboraci\'on SDSS. 

\end{itemize}

En los \'ultimos 5 a\~nos, el lider del grupo se ha dedicado a
investigar y publicar en el tema. En los temas relevantes para esta
propuesta sobresalen las siguientes publicaciones:

\noindent
En el tema de simulaciones y evoluci\'on de estructura a gran escala:

\begin{itemize}

\item {\it Cosmic web alignments with the shape, angular momentum and
  peculiar velocities of dark matter halos} {\bf Forero-Romero} J.E.,
  Contreras S., Padilla N., Accepted for publication in MNRAS,
  arXiv:1406.0508.

\item{\it The velocity shear tensor: tracer of halo alignment},
  Libeskind N., Hoffman Y., {\bf Forero-Romero} J.E., Gottloeber S.,
  Knebe A., Steinmentz M., Klypin A., MNRAS 428, 2489, 2013 

\item
{\it The dark matter assembly of the Local Group in constrained cosmological
  simulations of a $\Lambda$CDM universe} {\bf Forero-Romero} J.E., Hoffman Y., Yepes G., Gottl\"ober S.,
  Piontek R., Klypin A., Steinmetz M., 
MNRAS, 417, 1434, 2011

\item
{\it Halo based reconstruction of the cosmic mass density field}
Mun\~oz-Cuartas J. C., M\"uller V, {\bf Forero-Romero} J. E., MNRAS,
417, 1303, 2011  

\item
{\it A Dynamical Classification of the  Cosmic Web}.  {\bf
  Forero-Romero} J.E., Hoffman Y.,  Gottloeber S., Klypin A., Yepes G., MNRAS, 396, 1815-1824, 2009 

\item 
{\it The coarse geometry of merger trees in
  $\Lambda$CDM}.  {\bf Forero-Romero} J.E., 
MNRAS, 399, 762-768, 2009

\end{itemize}

\noindent
En el \'area de utilizar eventos extremos como prueba cosmol\'ogica:

\begin{itemize}
\item{\it The abundance of Bullet Groups in $\Lambda$CDM},
  J. G. Fern\'andez-Trincado, {\bf J. E. Forero-Romero}, G. Foex,
  V. Motta, T. Verdugo, V. Motta, ApJ Letter, 787, L32, 2014.
\item
{\it Bullet Clusters in the MareNostrum Universe}. 
{\bf Forero-Romero} J.E., Yepes G., Gottl\"ober S., 
ApJ, 725, 1, 2010.
\end{itemize}


\noindent 
En el \'area de analizar datos de simulaciones cosmol\'ogicas
para hacerlas disponibles a la comunidad acad\'emica:

\begin{itemize}
\item{\it The MultiDark Database: Release of the Bolshoi and
  MultiDark Cosmological Simulations} , K. Riebe , A. M. Partl,
  H. Enke, {\bf J.E. Forero-Romero}, S. Gottloeber, A. Klypin,
  G. Lemson, F. Prada, J. R. Primack, M. Steinmetz, V. Turchaninov,
  Astronomische Nachrichten, 334, 691, 2013. 
\end{itemize}


