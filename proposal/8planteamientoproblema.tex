

\section{Planteamiento del problema}
% Delimitación clara y precisa del objeto de la investigación que
% se realiza por medio de una pregunta.

La pregunta que dirige la investigaci\'on de este proyecto es: ¿Cu\'al
es la estructura del Universo en el que vivimos?  

Nuestro objetivo principal es crear diferentes universos virtuales
para estudiarlos como lo har\'ia un observador. Con esto podremos
encontrar una correspondencia entre rasgos observables y
par\'ametros o caracter\'isticas intr\'insicas de nuestro Universo.

En este trabajo partimos del modelo es\'tandar de la cosmolog\'ia
establecido en las \'ultimas dos d\'ecadas. Este modelo afirma que el
contenido de materia en el Universo est\'a dominado por la materia oscura. 
Adicionalmente incluye una componenten conocida como la constante
cosmol\'ogica  que explica la expansi\'on acelerada del Universo. 
En este modelo la materia bari\'onica es una fracci\'on minoritaria
del contenido de materia ener\'gia del Universo. 
La repartici\'on de estas tres componentes en t\'erminos de fracciones
de la densidad total correponden aproximadamente a un $5\%$ para los
bariones, un $25\%$ para la materia oscura y un $70\%$ para la
constante cosmol\'ogica. Adicionalmente, en este modelo consideramos
que la teor\'ia que describe la interacci\'on gravitacional es la
Relatividad General de Einstein. 

En este contexto, {\bf las simulaciones computacionales son la mejor
herramienta} para seguir la evoluci\'on temporal de la distribuci\'on
de materia. 
Este gran avance en los m\'etodos computacionales ha estado motivado por
los avances en t\'ecnicas observacionales que permiten hacer mapas del
Universo a grandes escalas a partir de mediciones de los espectros de
millones de galaxias.   
Actualmente, estas campa\~nas observacionales tambi\'en se
retroalimentan de los resultados de las simulaciones. 
Esto hace que finalmente las simulaciones computacionales sean una
herramienta \'util para los astrof\'isicos te\'oricos y los
astr\'onomos observacionales. {\bf Sin el trabajo conjunto de
  simuladores y observadores ser\'ia imposible inferir la estructura
  que tiene el Universo.} 

Nuestro proyecto busca dar respuesta a la pregunta sobre la
estructura del Universo realizando simulaciones para estudiar posibles
efectos medibles y as\'i mismo aplicar este conocimiento en la
planeaci\'on de observaciones con instrumentos de siguiente
generaci\'on.  

