
\section{Resumen ejecutivo}
% Información mínima necesaria para comunicar de manera precisa los
% contenidos y alcances del proyecto. 

La {\bf cosmología observacional} entró en una época dorada con la medición
de las anisotropías de la radiación cósmica de fondo (Premio Nobel de
Física 2006) y de la expansión acelerada del Universo
(Premio Nobel de Física 2011).  
Hoy en d\'ia una de las fronteras de la investigación en esta \'area 
es la {\bf observaci\'on del Universo a gran escala} para inferir la
historia de expansión  del Universo.   

El objetivo de esta \'area de investigaci\'on es entonces crear
grandes mapas tridimensionales del Universo observable. 
Una de las t\'ecnicas observacionales que se utiliza con ese
prop\'osito es la detecci\'on del pico de Oscilaciones Ac\'usticas de
Bariones (OAB) que requiere la medici\'on de las \emph{posiciones} de
millones de galaxias \cite{Eisenstein2005}. 
Otros m\'etodos usan informaci\'on sobre las \emph{velocidades}
peculiares (i.e. velocidades que no es\'an asociadas al flujo de
Hubble) de las galaxias y las distorsiones que estas generan en las
observaciones para cuantificar el crecimiento de estructura a gran
escala \cite{Scoccimarro2004}. 
Las {\bf simulaciones computacionales}
juegan un rol central en estos 
esfuerzos observacionales para medir el Universo.
Ellas son necesarias para traducir las premisas te\'oricas
en observables. 
Es decir, son un puente entre la teor\'ia y la  observaci\'on. 
Adem\'as sirven para dise\~nar y preparar 
nuevas campa\~nas observacionales.  
El objetivo es simular todo antes de empezar a medir. 


{\bf La presente propuesta tiene como objetivo principal entender, a
  trav\'es de simulaciones, los l\'imites observaciones para 
  medir la estructura del Universo a gran escala.}
  Este trabajo se har\'a principalmente dentro del marco de la
colaboraci\'on internacional DESI (Dark Energy Spectroscopic
Instrument).  DESI es una colaboraci\'on internacional de m\'as de 100
cient\'ificos que construir\'a un experimento de siguiente
generaci\'on para medir la historia de expansi\'on del Universo
haciendo un mapa de la  distribuci\'on de 25 millones de galaxias, 10
veces m\'as de lo que ha sido observado hasta la fecha \cite{DESI}. 

Alcanzar nuestro objetivo principal implica desarrollar nuevo software
y procesar datos existentes para estudiar fen\'omenos cosmol\'ogicos 
dentro de universos virtuales.
En este proceso utilizaremos el cluster de 480 procesadores de
las Instalaciones para C\'omputo de Alto Rendimiento de  la
Universidad de los Andes que fueron puestas en funcionamiento a
comienzos del 2015, as\'i como infraestructura computacional com\'un a
la colaboraci\'on DESI.
