
\section{Trayectoria del equipo de investigaci\'on}
%Incluir el estado actual de investigación del equipo que conforma la
%propuesta, así como las perspectivas de investigación dentro de la
%temática enmarcada en el proyecto propuesto. 

Actualmente el equipo de investigaci\'on se encuentra compuesto por
Dr. Jaime E. Forero-Romero y el estudiante de doctorado Felipe
G\'omez. Hay dos estudiantes que empezar\'an su doctorado en la
Universidad de los Andes en el tema de cosmolog\'ia durante el primer
semestre del 2015. Uno de ellos estar\'a involucrado en este proyecto.  

Las investigaciones actuales de Felipe G\'omez est\'an centradas en el
an\'alisis de simulaciones cosmol\'ogicas como trabajo
preparatorio. El tema central de su trabajo actual es la estimaci\'on
de la relaci\'on entre masa de materia oscura y taza de formaci\'on
estelar para galaxias a alto redshift. Esto va a ser \'util al momento
de generar cat\'alogos ficticios de galaxias a partir de las
simulaciones.  

En el caso de Jaime Forero se encuentra trabajando en la parte de
distribuci\'on de fibras \'opticas. El siguiente art\'iculos en
colaboraci\'on con cient\'ificos de DESI se encuentra enviado para una
colecci\'on de proceedings:\\ 


{\it Target allocation yields for massively multiplexed
  spectroscopic surveys with fibers}, Saunders W., Smedley S.,
  Gillingham P., {\bf Forero-Romero J.E.}, Jouvel S., Nord B., 
  SPIE 2014 9150-78, arXiv:1406.1787.


En paralelo la cooperaci\'on con KIAS en Corea del Sur ha avanzado de
tal manera a tener acceso a cat\'alogos ficticios construidos con la
simulaci\'on Horizon Run y empezar a trabajar sobre el
Alcock-Paczynski test. 

En cuanto a eventos extremos como prueba de LCDM, recientemente el
siguiente paper fue publicado.\\ 

{\it The abundance of Bullet Groups in $\Lambda$CDM},
  J. G. Fern\'andez-Trincado, {\bf J. E. Forero-Romero}, G. Foex,
  V. Motta, T. Verdugo, V. Motta, ApJ Letter, 787, L32, 2014.\\

En la Universidad de los Andes la compra de los equipos de c\'omputo
en paralelo ya empez\'o para tenerlos funcionando a finales del
2014. Entretanto hay una tesis de pregrado de F\'isica que se centra
en hacer simulaciones de estructura a gran escala con diferentes
cosmolog\'ias y una resoluci\'on baja ($512^3$ part\'iculas) como
preparaci\'on del trabajo del proyecto \coco. 


