
\section{Metodolog\'ia}
\label{metodologia}
% Exposición en forma organizada y precisa de cómo se desarrollará y
% alcanzará el objetivo general y cada uno de los objetivos
% específicos del proyecto, presentando los componentes del mismo y
% las actividades para el logro de estos. 

Para lograr los objetivos de este proyecto hemos asociado a cada uno
de los objetivos espec\'ificos una serie de tareas que deben ser
completadas. Jaime E. Forero Romero actuar\'a como coordinador de cada
una de estas tareas y seguir\'a su desarrollo y cumplimiento. 

Cada una de estas tareas tiene responsables para su ejecuci\'on. Estos
responsables ser\'an:

\begin{itemize}
\item Dr. Jaime E. Forero Romero  \prof.
\item Felipe G\'omez, estudiante de doctorado \gradA.
\item Estudiante de doctorado por determinar \gradB.
\item Asistentes t\'ecnicos de Uniandes \tecn.
\end{itemize}

A continuaci\'on relacionamos cada uno de los objetivos espec\'ificos
con la lista de tareas y sus responsables. 

\subsection*{T1. Infraestructura computacional para procesar simulaciones cosmol\'ogicas en Uniandes.} 
\begin{itemize}
\item[T1.1] \tecn 
Compra, instalaci\'on y mantenimiento de 12TB espacio de disco
  con backup para almacenar los datos originales de las simulaciones.
\item[T1.2] \tecn 
  Compra, instalaci\'on y mantenimiento de un blade de
  procesamiento con 24 procesadores con 512GB para poder analizar y
  postprocesar las simulaciones.
\item[T1.3] \tecn\prof 
  Transferir datos y sofware de la colaboraci\'on
  DESI a las m\'aquinas de Uniandes para hacer pruebas de todos los
  pasos siguientes antes de pasar a una implementaci\'on definitiva en
  los clusters computacionales de la colaboraci\'on DESI. 
\end{itemize}

\subsection*{T2. Analizar relaci\'on entre materia oscura y luminosidad para galaxias del tipo ELG, LRG y QSO}
\begin{itemize}
\item[T2.1] \gradA\prof 
  Escribir software para la generaci\'on de cat\'alogos
  ficticios de galaxias a partir de alguna relaci\'on anal\'itica
  entre masa de halo de materia oscura y luminosidad.
\item[T2.2] \gradA\prof 
  Escribir software que haga lo comparaci\'on de los cat\'alogos
  anteriores con datos observacionales.
\end{itemize}

\subsection*{T3. Encontrar los mejores par\'ametros de modelos
  te\'oricos para generar cat\'alogos ficticios de ELG, LRG y QSO.}
\begin{itemize}
\item[T3.1] \gradA\prof  
  Explorar el espacio de par\'ametros 
\item[T3.2] \gradA\prof  
  Escribir software que haga lo comparaci\'on
  de los cat\'alogos anteriores con datos observacinales.
\end{itemize}


\subsection*{T4. Preparar cat\'alogos ficticios de galaxias}
\begin{itemize}
\item[T4.1] \gradA\prof Escribir software para la generaci\'on de cat\'alogos
  ficticios de galaxias aleatorios (i.e. sin ning\'un tipo de
  clustering). 
\item[T4.2] \gradA\prof Adaptar el c\'odigo {\texttt
  make\_survey}\footnote{\url{https://github.com/mockFactory/make_survey}}
  para generar  cat\'alogos fictios de galaxias a partir de
  cat\'alogos de halos materia oscura. 
\end{itemize}

\subsection*{T5. Secuencias de observaci\'on para el experimento DESI}
\begin{itemize}
\item[T5.1] \prof Escribir software que simule la secuencia de observaci\'on
  de DESI sobre un cat\'alog de galaxias generado aleatoriamente. 
\item[T5.2] \prof Escribir software que simule la secuencia de observaci\'on
  de DESI sobre un cat\'alogo de galaxias generado a partir de una
  simulaci\'on de N-cuerpos. 
\end{itemize}

\subsection*{T6. Distribuci\'on de fibras \'opticas en el experimento DESI}
\begin{itemize}
\item[T6.1] \prof Escribir software que simule la asignaci\'on de fibras
  \'opticas sobre un cat\'alog de galaxias generadas aleatoriamente.  \bob.
\item[T6.2] \prof Escribir software que simule la asignaci\'on de fibras
  \'opticas sobre un cat\'alogo de galaxias generado a partir de una
  simulaci\'on de N-cuerpos. 
\end{itemize}

\subsection*{T7. Probabilidad de medir un redshift}
\begin{itemize}
\item[T7.1] \prof Escribir sofware que calcule la probabilidad de
  medir con \'exito un redshift con \'exito dependiendo del tipo de
  galaxia, su luminosidad, las condiciones del cielo y el instrumento.
\end{itemize}

\subsection*{T8. Simulacion de un cat\'alogo que incluye la
  probabilidad de medir un redshift.}
\begin{itemize}
\item[T8.1] \prof Escribir software que calcule un cat\'alogo de galaxias
  generado a partir de una simulaci\'on de N-cuerpos teniendo en
  cuenta las diferentes probabilidades de medir un redshift.
\end{itemize}

\subsection*{T9. Integrar el software de todos los pasos anteriores}
\begin{itemize}
\item[T9.1] \gradB\prof Integrar en la base de c\'odigo de DESI el software para
  preparar cat\'alogos ficticios de galaxias a partir de simulaciones
  de N-cuerpos. 
\item[T9.2] \gradB\prof Integrar en la base de c\'odigo de DESI el software para
  definir las secuencias de observaci\'on del experimento.

\item[T9.3] \gradB\prof Integrar en la base de c\'odigo de DESI el software para
  definir la asignaci\'on de fibras \'opticas.
\item[T9.4] \gradB\prof Integrar el c\'odigo completo de simulaci\'on end-to-end
  de DESI. 
\end{itemize}

\subsection*{T10. Producir cat\'alogos de galaxias con todos los
  efectos instrumentales de DESI}
\begin{itemize}
\item[T10.1] \gradB\prof Producir simulaciones completas de
  simulaci\'on end-to-end de DESI. Desde las observaciones hasta la
  estimaci\'on de redshifts de galaxias observadas. 
\end{itemize}

\subsection*{T11. Efectos instrumentales sobre mediciones OAB}
\begin{itemize}
\item[T11.1] \gradB Cuantificar el efecto instrumental sobre
  mediciones de Oscilaciones Ac\'usticas de Bariones usando
  cat\'alogos de galaxias simlados.
\end{itemize}

\subsection*{T12. Efectos instrumentales sobre mediciones DER}
\begin{itemize}
\item[T12.1] \gradB Cuantificar el efecto instrumental sobre
  mediciones de Distorsiones en el Espacio de Redshift
  cat\'alogos de galaxias simlados.
\end{itemize}


