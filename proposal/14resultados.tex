
\section{Resultados esperados de la investigaci\'on}
% Conocimiento generado en el cumplimiento de cada uno de los
% objetivos. 

Al final del proyecto se debe haber generado nuevo conocimiento sobre las siguientes puntos 
 
\begin{itemize}
\item Utilizaci\'on de sistemas de c\'omputo masivamente para la
  realizaci\'on de simulaciones de estructura del Universo a gran
  escala.
\item Entendimiento del impacto de la repartici\'on de fibras
  \'opticas y de la estrategia de observaci\'on del experimento DESI
en su capacidad para acotar los par\'ametros cosmol\'ogicos.
\item Cuantificaci\'on de la abundancia de colisiones extremas entre
  pares de halos de materia oscura bajo diferentes cosmolog\'ias de la
  familia $w$CDM.
\item Cotas de los par\'ametros cosmol\'ogicos a trav\'es de pruebas
  que utilizan el test de Alcock-Paczynski para cuantificar la
  distribuci\'on anisotr\'opica de galaxias a gran escala.
\item Metodolog\'ias para analizar datos de la distribuci\'on de
  galaxias a gran escala y obtener informaci\'on de relevancia
  cosmol\'ogica a partir de las distorsiones en el espacio de
  redshift.
\end{itemize}

