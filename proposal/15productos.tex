
\section{Resultados esperados - Productos}
% Evidencian el logro en cuanto a generación de nuevo conocimiento,
% fortalecimiento de capacidades científicas y apropiación social del
% conocimiento, incluir indicadores verificables y medibles acordes
% con los objetivos y alcance del proyecto. 

%(Productos, haciendo particular énfasis en los asociados con
%generación de nuevo conocimiento, con el fortalecimiento de
%capacidades científicas y con la apropiación social del
%conocimiento). Se deben considerar los productos relacionados en el
%documento conceptual que hace parte de la Medición de Grupos de
%Investigación, Desarrollo Tecnológico y/o Innovación, 20132 (ver
%anexo 10). 

A continuaci\'on relacionamos los resultados esperados en términos de productos 

\begin{itemize}
\item Avances de una tesis doctoral en temas relacionados con la
  planeaci\'on de DESI. 

\item Avances de una tesis doctoral en temas relacionados con
  estimaci\'on de par\'ametros cosmol\'ogicos a partir de datos
  observacionales. 

\item Presentaci\'on de resultados en al menos dos eventos
  internacionales.  

\item Presentaci\'on de resultados en el congreso colombiano de
  f\'isica en el 2015.   

\item Presentaci\'on de resultados en el congreso colombiano de
  astrof\'isica en el 2016.  

\item Publicaci\'on de al menos dos art\'iculos en revistas
  internacionales indexadas. 

\item Producci\'on de al menos un producto de sofware para la
  simulaci\'on de instrumentos de espectroscop\'ia masiva.  

\item Organizar una escuela internacional de cosmolog\'ia
  computacional, Junio del 2015, en la Universidad de los Andes.  

\item Creaci\'on de una p\'agina web formato weblog dedicada a mostrar
  los avances del proyecto. 

\item Creaci\'on de una p\'agina web para hacer disponibles los datos
  de las simlulacioes a la comunidad astron\'omica nacional e
  internacional. 

\item Al menos dos charlas p\'ublicas de divulgaci\'on sobre los temas
  de cosmolog\'ia computacional y observacional. 



\end{itemize}
