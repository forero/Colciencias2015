
\section{Resultados esperados - Productos}
% Evidencian el logro en cuanto a generación de nuevo conocimiento,
% fortalecimiento de capacidades científicas y apropiación social del
% conocimiento, incluir indicadores verificables y medibles acordes
% con los objetivos y alcance del proyecto. 

%(Productos, haciendo particular énfasis en los asociados con
%generación de nuevo conocimiento, con el fortalecimiento de
%capacidades científicas y con la apropiación social del
%conocimiento). Se deben considerar los productos relacionados en el
%documento conceptual que hace parte de la Medición de Grupos de
%Investigación, Desarrollo Tecnológico y/o Innovación, 20132 (ver
%anexo 10). 

A continuaci\'on relacionamos los resultados esperados en términos de productos 

\begin{enumerate}

\item Avances de una tesis doctoral en temas relacionados con
  la relaci\'on entre halos de materia oscura y propiedades
  observables de una galaxia. 
El estudiante vinculado a este punto 
  es Felipe Leonardo G\'omez Cort\'es
  \url{http://scienti1.colciencias.gov.co:8081/cvlac/visualizador/generarCurriculoCv.do?cod_rh=0001531135}  

\item Presentaci\'on de resultados en al menos dos eventos
  internacionales.  

\item Presentaci\'on de resultados en al menos dos eventos nacionales.

\item Publicaci\'on de al menos tres art\'iculos en revistas
  internacionales indexadas. 

\item Producci\'on de al menos un producto de sofware para la
  simulaci\'on de instrumentos de espectroscop\'ia masiva.  

\item Creaci\'on de una p\'agina web formato weblog dedicada a mostrar
  los avances del proyecto. 

\item Al menos dos charlas p\'ublicas de divulgaci\'on sobre los temas
  de cosmolog\'ia computacional y observacional. 

\end{enumerate}
